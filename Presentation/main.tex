\documentclass{article}
\usepackage[english]{babel}
\usepackage[utf8x]{inputenc}
\usepackage[T1]{fontenc}
\usepackage{graphicx}
\usepackage[colorinlistoftodos]{todonotes}
\usepackage[colorlinks=true, allcolors=tudelftblue]{hyperref} %sets hyperlink colour
\usepackage{caption}
\usepackage{subcaption}
\usepackage{xcolor}
\usepackage{roboto} % for Roboto Slab font
\usepackage{float}
\usepackage{titling} 
\usepackage{blindtext}\usepackage{titlesec}
\usepackage[square,sort,comma,numbers]{natbib}
\usepackage[colorinlistoftodos]{todonotes}
\usepackage{tikz}
\usepackage{geometry}
\usepackage{sectsty}
\usepackage{amsmath}
\usepackage{tikzpagenodes}
\usepackage{booktabs}
\usepackage{listings}
\usepackage{attachfile}
\definecolor{tudelftdarkblue}{RGB}{12,35,64}
\definecolor{tudelftcyan}{RGB}{0,166,214}
\definecolor{tudelftblue}{RGB}{0, 118, 194}
\geometry{a4paper, margin=2cm}
\allsectionsfont{\color{black}} %sets colour for all headers
\usepackage{helvet}
\renewcommand{\familydefault}{\sfdefault}
\sectionfont{\fontfamily{RobotoSlab-TLF}\selectfont}
%%%%%%%%%%%%%%%%%%%%%%%%%%%%%%%%%%%%%%%%%%%%%%%%%%%%%%%%%
\begin{document}

\input{titlepage}

%%% Create a table of contents
\tableofcontents
\newpage

\addcontentsline{toc}{section}{Introduction}
\section{Introduction}
\subsection{Background and Motivation}
Survival analysis is a branch of statistics that deals with the analysis of time-to-event data, commonly encountered in various fields such as biomedical research, engineering, and social sciences. Traditional methods of survival analysis have been extensively used to analyze censored data and estimate survival probabilities. However, these methods often rely on assumptions that may not hold in complex real-world scenarios, where non-linear relationships and high-dimensional data are common.

In recent years, the advent of machine learning has brought new opportunities to the field of survival analysis. Machine learning methods offer flexible and powerful alternatives to traditional approaches, including survival trees, random survival forests, support vector regression, and neural networks. These methods are capable of handling complex interactions, capturing non-linear relationships, and integrating high-dimensional covariates without the need for strict parametric assumptions. As a result, they have gained significant attention for their potential to improve predictive accuracy and provide more comprehensive insights into survival data.

\subsection{Objectives}
This project aims to explore and compare several machine learning methods for survival analysis. Through a combination of simulation studies and real data applications, we seek to evaluate the performance of these methods in various settings. The objectives of this study are to understand the strengths and limitations of each method, implement them using statistical software such as Python and R, and provide practical recommendations for their use in different contexts. By doing so, we hope to contribute to the growing body of knowledge on machine learning techniques in survival analysis and offer valuable tools for researchers and practitioners in the field.

\section{Overview of Survival Analysis}
Survival Analysis is a branch of statistics that deals with the analysis of time-to-event data. The primary focus is on modeling and analyzing the time until an event of interest occurs

-censoring: left, interval, right, truncation: left, right:
Right censoring is the most common type, where the event has not occurred by the end of the study or the subject is lost to follow-up.Left censoring occurs when the event of interest has already occurred before the study begins. Interval censoring happens when the event is known to have occurred within a specific time interval, but the exact time is unknown.

-survival function, S(t): probability that an individual survives beyond a certain time point. The survival function is a non-increasing function over time, starting at 1 when t=0t=0 and approaching 0 as tt goes to infinity.

\begin{equation*}
    S(t) = P(T>t) = e^{-H(t)}
\end{equation*}

Kaplan-Meier curves: non-parametric estimate survival probabilities over time when censoring is present, calculation, confidence interval(log, log-log, plain)

-hazard function, h(t): instantaneous risk of an event at a given time. describes the instantaneous rate at which events occur at time t, given that the individual has survived up to time t.
\begin{equation*}
    h(t) = \lim_{\Delta t \to 0} \frac{P(t \leq T < t + \Delta t|T \geq t)}{\Delta t} = \lim_{\Delta t \to 0} \frac{F(t + \Delta t)-F(t)}{S(t)\Delta t } = \frac{f(t)}{S(t)} = -\frac{d}{dt}S(t) \cdot \frac{1}{S(t)} = -\frac{d}{dt}\left[\log S(t)\right]
\end{equation*}

\begin{equation*}
    H(t) = \int^t_0 h(u)du
\end{equation*}

-Log Rank Test: testing subpopulations are significantly different

-chi-square test

-Cox Proportional Hazards Model: A semi-parametric model that relates the hazard function to explanatory variables

"For a given instance i, represented by a triplet $(X_i , y_i, \sigma_i )$, the hazard function $h(t, X_i)$ in the Cox model follows the proportional hazards assumption given by
\begin{equation*}
    h(t, X_t) = h_0(t)e^{X_t\beta}
\end{equation*}

for $i = 1, 2, \dots, N$ , where the baseline hazard function, $h_0(t)$, can be an arbitrary non-negative function of time; $X_i = (x_{i1}, x_{i2}, \dots , x_{ip})$ is the corresponding covariate vector for instance $i$; and $\beta^T = (\beta_1, \beta_2, \dots , \beta_p)$ is the coefficient vector. The Cox model is a semi-parametric algorithm since the baseline hazard function, $h_0(t)$, is unspecified."

-life tables


\section{Methodology}
\subsection{Description of Machine Learning Methods}
-Survival Trees:
"The decision tree method performs recursive partitioning on the data by setting a threshold for each feature, but it can neither consider the interactions between the features nor the censored information in the model\cite{SafavianLandgrebe}. The splitting criteria used for survival trees can be grouped into two categories; namely, those that minimize within-node homogeneity and those that maximize between-node heterogeneity."\cite{WangEtAl}

Survival trees can handle right-censored data. 

Survival trees are constructed by recursively partitioning the data into subsets that are increasingly similar with respect to the survival outcome. At each node of the tree, the data is split based on a set of covariates that best differentiate between different survival times.

Each leaf of the tree represents a subgroup of subjects with similar survival characteristics.

-Random Survival Forests:

 ensemble method that extends survival trees by combining multiple trees to improve accuracy and control overfitting.

Construct multiple survival trees using random subsets of the data and a random subset of features for each split
 
Aggregate survival predictions from all trees to compute the overall survival function or cumulative hazard function

-Support Vector Machine

-Neural Networks: https://k-d-w.org/blog/2019/07/survival-analysis-for-deep-learning/

Uses layers of neurons to learn hierarchical representations of input data

Train the model using backpropagation and an appropriate survival loss function

\subsection{Implementation}
-Software and tools used (Python, R, relevant libraries and packages).

-Data preprocessing and handling of censored data.

\subsection{Simulation Studies}
-Design of simulation experiments to evaluate the performance of different methods.

-Metrics used for comparison (e.g., Concordance index, Integrated Brier Score).

\section{Real Data Analysis}
\subsection{Description of Data}
2509 Subjects

34 columns:
\begin{table}[]
\begin{tabular}{c|c||c|c}
\multicolumn{1}{l}{\textbf{Covariate}} & \% Null  & \multicolumn{1}{l}{\textbf{Covariate}} & \% Null  \\
\hline
Patient ID                             & 0.000000 & Integrative Cluster                    & 0.210841 \\
Age at Diagnosis                       & 0.004384 & Primary Tumor Laterality               & 0.254683 \\
Type of Breast Surgery                 & 0.220805 & Lymph nodes examined positive          & 0.106018 \\
Cancer Type                            & 0.000000 & Mutation Count                         & 0.060582 \\
Cancer Type Detailed                   & 0.000000 & Nottingham prognostic index            & 0.088481 \\
Cellularity                            & 0.235951 & Oncotree Code                          & 0.000000 \\
Chemotherapy                           & 0.210841 & Overall Survival (Months)              & 0.210442 \\
Pam50 + Claudin-low subtype            & 0.210841 & Overall Survival Status                & 0.210442 \\
Cohort                                 & 0.004384 & PR Status                              & 0.210841 \\
ER status measured by IHC              & 0.033081 & Radio Therapy                          & 0.210841 \\
ER Status                              & 0.015943 & Relapse Free Status (Months)           & 0.048226 \\
Neoplasm Histologic Grade              & 0.048226 & Relapse Free Status                    & 0.008370 \\
HER2 status measured by SNP6           & 0.210841 & Sex                                    & 0.000000 \\
HER2 Status                            & 0.210841 & 3-Gene classifier subtype              & 0.296931 \\
Tumor Other Histologic Subtype         & 0.053806 & Tumor Size                             & 0.059386 \\
Hormone Therapy                        & 0.210841 & Tumor Stage                            & 0.287365 \\
Inferred Menopausal State              & 0.210841 & Patient's Vital Status                 & 0.210841
\end{tabular}
\end{table}

\subsection{Application of Machine Learning Methods}

\section{Results and Discussion}
\subsection{Simulation Results}
-Detailed presentation of results from simulation studies.

-Comparison of methods based on simulation outcomes.

\subsection{Real Data Results}
-Analysis of real data results and insights gained.

\subsection{Discussion}
-Interpretation of findings.

\section{Conclusion}
-Summary

\newpage
\addcontentsline{toc}{section}{References}
\bibliographystyle{apalike}
\bibliography{ref}

\newpage
\appendix



\end{document}






%All other official TU Delft colours
\definecolor{donkerblauw}{RGB}{12, 35, 64}
\definecolor{turkoois}{RGB}{0, 184, 200}
\definecolor{blauw}{RGB}{0, 118, 194}
\definecolor{paars}{RGB}{111, 29, 119}
\definecolor{roze}{RGB}{239, 96, 163}
\definecolor{framboos}{RGB}{165, 0, 52}
\definecolor{rood}{RGB}{224, 60, 49}
\definecolor{oranje}{RGB}{237, 104, 66}
\definecolor{geel}{RGB}{255, 184, 28}
\definecolor{lichtgroen}{RGB}{108, 194, 74}
\definecolor{donkergroen}{RGB}{0, 155, 119}
%You can use these to change the hyperlink colour, the colour of the header or whatever. Glück Auf!